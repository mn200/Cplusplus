\documentclass[11pt]{article}

\usepackage{charter}
\usepackage{alltt}
\usepackage{url}
\usepackage{proof}
\usepackage{amsfonts}
\usepackage{underscore}
\usepackage{pstricks}
\usepackage{pst-node}

\newcommand{\cpp}{\mbox{C\hspace{-0.5pt}+\hspace{-1.5pt}+}}
\newcommand{\ruleid}[1]{\textsl{[#1]}}

\newcommand\ckey[1]{\texttt{#1}}

\newcommand\statebrack[1]{\langle #1 \rangle}
\newcommand\exprarrow{\rightarrow_e}
\newcommand\tcexarrow{\exprarrow^*}
\newcommand\expreval[2]{\statebrack{#1} \exprarrow \statebrack{#2}}
\newcommand\tcexeval[2]{\statebrack{#1} \tcexarrow \statebrack{#2}}
\newcommand\stmtarrow{\rightarrow_s}
\newcommand\stmteval[2]{\statebrack{#1} \stmtarrow \statebrack{#2}}
\newcommand\vardarrow{\rightarrow_v}
\newcommand\vardeval[2]{\statebrack{#1}\vardarrow \statebrack{#2}}
\newcommand\hoaretriple[3]{\{#1\}\;#2\;\{#3\}}
\newcommand\hoarequad[4]{\{#1\}\;#2\;\{#3\}\{#4\}}

\newcommand\inferwhere[1]{#1\par\nopagebreak\vskip 1mm\nopagebreak%
                          \noindent\ignorespaces}

\newcommand\okval{\ckey{okVal}}
\newcommand\tyobj[1]{\operatorname{\textbf{obj}}[#1]}

\newcommand\addse{\ckey{add\_se}}
\newcommand\applyse{\ckey{apply\_se}}
\newcommand\clearse{\ckey{clear\_se}}
\newcommand\declvar{\ckey{decl\_var}}
\newcommand\initdeclvar{\ckey{init\_decl\_var}}
\newcommand\installparams{\ckey{inst\_parms}}
\newcommand\markref{\ckey{mark\_ref}}
\newcommand\memtrans{\ckey{mem\_trans}}
\newcommand\removerefs{\ckey{remove\_refs}}
\newcommand\structdeclare{\ckey{struct\_decl}}

\newcommand\cid{{\sf id}}
\newcommand\cgenop{\odot}
\newcommand\Cand{\ckey{\&\&}}
\newcommand\Cor{\ckey{||}}
\newcommand\ccond[3]{#1 \;\ckey{?}\; #2 \;\ckey{:}\; #3}
\newcommand\cassign[2]{#1\mathbin{\ckey{=}}#2}
\newcommand\ccast[2]{\ckey{(}#1\ckey{)}#2}
\newcommand\copassign[3]{#1\mathbin{#2\ckey{=}}#3}
\newcommand\andorsqpt{\widehat{\ckey{\&|}}}
\newcommand\cptr{\ptrtype}
\newcommand\se[2]{\clubsuit(#1,#2)}
\newcommand\rhsop{\diamondsuit}
\newcommand\ass[4]{%
  #2 \mathbin{\raisebox{-1.5mm}{$\stackrel{#4}{#1\ckey{=}}$}} #3}
\newcommand\cif[3]{%
\mbox{\ckey{if} \ckey{(}$#1$\ckey{)} $#2$ \ckey{else} $#3$}}
\newcommand\cloop{\mbox{$O$}}
\newcommand\trap{\mbox{$T$}}

\newcommand\declstmts{\mbox{\sl decls}\;\mbox{\sl stmts}}
\newcommand\cblock[1]{\mbox{\bf \{} #1\mbox{\bf \}}}
\newcommand\cwhile[2]{\mbox{\ckey{while} \ckey{(}$#1$\ckey{)} $#2$}}
\newcommand\cfor[4]{\mbox{\ckey{for ($#1$; $#2$; $#3$) $#4$}}}
\newcommand\cdowhile[2]{\mbox{\ckey{do $#1$ while ($#2$);}}}
\newcommand\tradwhile[2]{\mbox{\bf while}\;#1\;\mbox{\bf do}\;#2}
\newcommand\tradif[3]{\mbox{\bf if}\;#1\;\mbox{\bf then}\;#2\;%
                      \mbox{\bf else}\;#3}

\newcommand\ptrtype[1]{#1\ckey{*}}
\newcommand\arraytype[2]{#1\ckey{[}#2\ckey{]}}
\newcommand\sgndtype[1]{\ckey{signed}\;#1}
\newcommand\usgndtype[1]{\ckey{unsigned}\;#1}
\newcommand\void{\ckey{void}}
\newcommand\longdouble{\ckey{long~double}}
\newcommand\structtype[1]{\ckey{struct}\;#1}
\newcommand\struct{\ckey{struct}}

\newcommand\cnum{\ckey{Num}\,}
\newcommand\cfunref{\ckey{Funref}\,}
\newcommand\cnull{0}
\newcommand\cvar{\ckey{Var}\,}
\newcommand\caddr[1]{\ckey{\&}#1}
\newcommand\cderef{\ckey{*}\,}
\newcommand\cbreak{\ckey{break}}

\newcommand\cholundef{{\cal U}}
\newcommand\cvalue[2]{\underline{(#1,#2)}}
\newcommand\clvalue{\ckey{LV}}
\newcommand\cfvalue{\ckey{FV}}
\newcommand\crvreq{\ckey{RVR}}


\newcommand\retval{\ckey{RetVal}}
\newcommand\stmtval{\ckey{StmtVal}}
\newcommand\breakval{\ckey{BreakVal}}
\newcommand\contval{\ckey{ContVal}}

\newcommand\vardval{\ckey{VarDeclVal}}


\newcommand\wftype[2]{#1\vdash \text{\emph{wf}}(#2)}
\newcommand\etyping[3]{#1\vdash #2 : #3}

\newcommand\ctilde{\ckey{\~{}}}

\newcommand\lvtorv{\leadsto}
\newcommand{\funtype}[2]{#1 \rightarrow #2}
\newcommand{\SOME}[1]{\lfloor #1 \rfloor}

\newcommand{\last}{\textsf{last}}
\newcommand{\hd}{\textsf{hd}}

\newenvironment{proof}{\sl\setlength\parindent{0mm}\addtolength\parskip{2mm}}%
{}

% Local Variables:
% mode: latex
% TeX-master: t
% End:


\newcommand{\ie}{\emph{i.e.}}
\newcommand{\naive}{na\"\i{}ve}

\title{Notes on Deliverable 4 (April 2007)}
\author{Michael Norrish\\{\small \texttt{Michael.Norrish@nicta.com.au}}}
\date{}


\begin{document}
\maketitle

\section{Form}

This deliverable consists of a compressed \texttt{tar}-file, that when
unpacked consists of a directory called \texttt{qinetiq-cpp}, which in
turn contains four directories
\begin{itemize}
\item \texttt{holsrcs}, containing the HOL source files of the
  mechanisation.  These files will build with the version of HOL4
  present in the CVS repository at SourceForge, with timestamp
  \texttt{2007-04-01 00:00Z}.  See Section~\ref{sec:getting-hol}
  below for instructions on how this version of HOL can be retrieved,
  and how the deliverable's HOL source files can then be built and
  checked.
\item \texttt{talks}, containing the \LaTeX{} source and a PDF for the
  talk presented at the DARP meeting in Newcastle in April 2006.  The
  source assumes that the \texttt{latex-beamer} and \texttt{PSTricks}
  packages are available.
\item \texttt{docs}, containing \LaTeX{} sources and a PDF version of
  this document, as well as sources for the notes on the earlier
  deliverables (nos.~1--3).
\item \texttt{notes}, some \cpp{} source files that illustrate various
  aspects of \cpp{} behaviour.  An accompanying text file explains some
  of the behaviours.
\end{itemize}

\subsection{Building HOL Source-Files}
\label{sec:getting-hol}

\paragraph{Getting HOL From SourceForge}

To get a particular, dated, version of the HOL4 sources from the CVS
repository, one must first issue the command

{\small
\begin{verbatim}
   cvs -d:pserver:anonymous@hol.cvs.sourceforge.net:/cvsroot/hol login
\end{verbatim}
}

When prompted for a password, just press \texttt{Enter} to send a null
response.  The check-out of source code from SourceForge can now
proceed.  The source code fits into 60MB.  Issue the command

{\small
\begin{alltt}
   cvs \textit{repository-spec} co -D \textit{date} hol98
\end{alltt}
}

\noindent where \textit{\ttfamily repository-spec} is the string

{\small
\begin{alltt}
   -d:pserver:anonymous@hol.cvs.sourceforge.net:/cvsroot/hol
\end{alltt}
}

\noindent (also used in the \texttt{login} command), and where
\textit{\ttfamily date} is the desired date, best specified as an
ISO~8601 string enclosed in double-quotes.  For example,
\texttt{"2006-06-30 04:05Z"}.

Once a copy of the sources have been downloaded, further commands can
be used to update this copy to correspond to different dates.  As long
as such commands are issued from within the \texttt{hol98} directory,
the repository specification can be omitted.  The update command is

{\small
\begin{alltt}
   cvs update -d -D \textit{date}
\end{alltt}
}

\paragraph{Installing HOL} Once the sources have been downloaded, the
installation instructions from the page at
\url{http://hol.sourceforge.net} should be followed to build a copy of
HOL.  An installation of the Moscow~ML compiler (v2.01) will also be
required.

\paragraph{Building Deliverable Sources}
When HOL4 has been installed, the \texttt{Holmake} program (found in
the \texttt{hol98/bin} directory) can be run in the \texttt{holsrcs}
directory to create and check the logical theories.

\section{Content}

This deliverable adds a treatment of \cpp{} exceptions
(Section~\ref{sec:exceptions}) and templates
(Section~\ref{sec:templates}) to the existing semantics.  It also
fixes a variety of minor bugs.

\subsection{Exceptions}
\label{sec:exceptions}

As suggested in Deliverable~1, exceptions can indeed be modelled in a
way similar to the existing treatment of \texttt{return},
\texttt{break} and \texttt{continue}.  One difference is that
exceptions propagate further: the \texttt{return} ``value'' only
propagates up as far as a function call (within an expression).  In
contrast, an exception will continue to propagate up through the
call-stack until it hits a suitable handler.

\newcommand{\ethrow}{\texttt{throw}\ensuremath{_e}}
This much allows a preliminary sketch of the behaviour.  The
\texttt{throw} form is actually an expression, but we set things up so
that there is a statement-level version of \texttt{throw} as well, and
it will be this that propagates through statement syntax.  The
expression syntax is written \ethrow{}, and the rule
\ruleid{expression-throw-some} describes the behaviour when the
\texttt{throw} has an argument:
\[
\infer{
  \statebrack{\textsf{Ex}(\ethrow(e)), \sigma}
  \to
  \statebrack{\textsf{St}(\texttt{throw}(e),c), \sigma}}{}
\]
The variable $c$ represents the continuation that would normally
convert the result of the statement into a value to be inserted into a
containing expression tree.  Because \texttt{throw} values can't ever
turn into values until they initialise a handler, this $c$ can be
anything at all.

At the statement level, the \texttt{throw} form takes an extended
expression as an argument.  This evaluates its argument as one might
expect (rule \ruleid{throw-expr-evaluation})
\[
\infer{\statebrack{\texttt{throw}\;e_0,\sigma_0}\stmtarrow\statebrack{\texttt{throw}\;e,\sigma}}{\statebrack{e_0,\sigma_0}\to\statebrack{e,\sigma}}
\]
When a \texttt{throw}'s expression has been completely evaluated, we
have something that can then propagate upwards through the abstract
syntax of statements.  Because of the way the rules for loops and
\texttt{if}-statements work, their sub-statements are never executed
while still sub-statements.

We start by specifying how \texttt{throw}-statements traverse
\texttt{Block} values.  This propagation uses the rule
\ruleid{block-interrupted}:
\[
\infer{
  \statebrack{
    \mathbf{\{} \;[\,] \;(s\!::\!\mathit{sts})\,\mathbf{\}},
    \sigma
  }
  \stmtarrow
  \statebrack{\mathbf{\{} \;[\,]\; [s]\mathbf{\}},\sigma}}{
  \textsf{final\_stmt}(s) & s \neq \textsf{EmptyStmt} & \mathit{sts}
  \neq [\,]}
\]
That this is a block rule is indicated by the
$\mathbf{\{\,\}}$-delimiters.  The first argument inside the block on
both sides of the arrow is an empty list of declarations.  The
predicate \textsf{final\_stmt} is true of \texttt{throw} and
\texttt{return} statements with fully evaluated arguments, as well as
of \texttt{break}, \texttt{continue} and the \textsf{EmptyStmt} form.
The latter doesn't cause an interruption, so is excluded by the second
hypothesis to the rule.  The final hypothesis ensures that there isn't
an infinite loop on this rule.

A \naive{} version of the rule for exiting a block
(\ruleid{block-exit}) is
\[
\infer{
  \statebrack{\mathbf{\{}\;[\,]\;[s]\;\mathbf{\}},\sigma}
  \stmtarrow
  \statebrack{s,\sigma}
}{
  \textsf{final\_stmt}(s)
}
\]
If a block has a final statement as its only statement, then that
statement can be propagated out to the top-level of the abstract
syntax. We will return to this rule when we come to worry about the
interaction of exceptions and object lifetimes.

There is also rule \ruleid{trap-exn-passes} for exception statements
escaping the \texttt{Trap} form (which is used for handling
\texttt{continue} and \texttt{break}):
\[
\infer{
  \statebrack{
    \textsf{St}(\texttt{Trap}\;\mathit{tt}\;\mathit{exn}, c), \sigma
  }
  \to
  \statebrack{\textsf{St}(\mathit{exn},c), \sigma}
}{\textsf{is\_exnval}(\textsf{St}(\mathit{exn},c))}
\]

\medskip
Because exceptions arise from expressions, the statement level rules
need to acknowledge this possibility.  Thus, this rule for
\texttt{if} (\ruleid{if-exception}):
\[
\infer{
  \statebrack{\textsf{St}(\texttt{if}\;G\;t\;e, c), \sigma}
  \to
  \statebrack{\textsf{mk\_exn}(G,c),\sigma}
}{
  \textsf{is\_exnval(G)}
}
\]
where $\textsf{is\_exnval}(G)$ is true if $G$ is of the form
$\textsf{St}(\texttt{throw}(e), c_0)$, and $e$ has been fully
evaluated.  The function \textsf{mk\_exn} takes an exception value and
replaces its continuation information with something appropriate for
the level of the containing statement.  In this example, $c_0$ will be
replaced by $c$.  There is a similar rule for the standalone
expression and \texttt{return} forms, as well as for the
statement-level \texttt{throw} form itself.  (The rule
\ruleid{expression-throw-some} turns a \ethrow{} into a \texttt{throw}
statement immediately, without evaluating the argument.  When the
argument \emph{is} evaluated, it may itself cause an exception.)

Because exceptions can arise in variable declarations, there is also a
rule for handling these.  This is \ruleid{block-declmng-exception}:
\[
\infer{
  \statebrack{
    \textsf{St}(\mathbf{\{}\,(d\!::\!\mathit{ds})\;sts\,\mathbf{\}}, c),
    \sigma_0
  }
  \to
  \statebrack{
    \textsf{St}(\mathbf{\{}\,[\,]\;[\texttt{throw}(\mathit{ex})]\,\mathbf{\}}, c),
    \sigma
  }
}{
  \statebrack{d,\sigma_0}\to_d
  \statebrack{[\texttt{VDecInitA}\;\tau\;\mathit{loc}\;e], \sigma}
  &
  e = \textsf{St}(\texttt{throw}(\mathit{ex}), c')
}
\]
Again, note how the continuation initially associated with the
exception ($c'$) is ignored.


\subsubsection{Handling Exceptions}

Handling exceptions is done with the \texttt{try-catch} form, which is a
sequence of handlers associated with a statement that might raise an
exception.  In the concrete syntax, programmers write something like
\newcommand{\suplus}{\ensuremath{^+}}
\newcommand{\sustar}{\ensuremath{^*}}
\begin{alltt}
   try \{
     \emph{statement}\sustar
   \}
   \emph{handler}\suplus
\end{alltt}
where a \emph{handler} is of the form
\begin{alltt}
   catch (\emph{guard}) \{ \emph{statement}\sustar \}
\end{alltt}
and a \emph{guard} can be ``\texttt{...}'' (\ie, three full-stops), a
type, or a standard parameter declaration (associating a type with a
name).

At the abstract syntax level, this is captured by the following HOL
declarations (in \texttt{statementsScript}):
\begin{alltt}
   exn_pdecl = (string option # CPP_Type) option

   stmt = ...
        | Catch of stmt => (exn_pdecl # stmt) list
\end{alltt}
In what is to come, I will write $*$ to correspond to the ``null''
value in option types, and use $\lfloor x\rfloor$ to make it explicit
that $x$ is in an option type.

\bigskip
\noindent
Statements can evaluate as usual under a \texttt{Catch}
\ruleid{catch-stmt-evaluation}:
\[
\infer{
  \statebrack{\textsf{St}(\texttt{Catch}\;s\;\mathit{hnds}, c),
    \sigma_0}
  \to
  \statebrack{\textsf{St}(\texttt{Catch}\;s'\;\mathit{hnds}, c),
    \sigma}
}{
  \statebrack{\textsf{St}(s,c),\sigma_0}\to
  \statebrack{\textsf{St}(s',c),\sigma}
}
\]
Non-exception statements pass through \texttt{Catch} statements,
ignoring the handlers \ruleid{catch-normal-stmt-passes}:
\[
\infer{
  \statebrack{
    \textsf{St}(\texttt{Catch}\;s\;\mathit{hnds}, c),
    \sigma
  }
  \to
  \statebrack{\textsf{St}(s,c), \sigma}
}{
  \textsf{final\_stmt}(s) &
  \lnot\textsf{is\_exnval}(\textsf{St}(s,c))
}
\]

There are three rules governing how handlers interact with thrown
exceptions.  The first describes the behaviour when the handler
parameter is given as ``\texttt{...}'' \ruleid{catch-all}:
\[
\infer{
  \begin{array}{l}
    \statebrack{
      \textsf{St}(\texttt{Catch}\;(\texttt{throw}(e))\;((*,
      \mathit{hnd})::\mathit{hnds}), c), \sigma
    }
    \to\\[1ex]
    \quad \statebrack{
      \textsf{St}(\mathbf{\{}\,[\,]\;[\mathit{hnd};\,\textsf{ClearExn}]\,\mathbf{\}}, c),
      \sigma[\textsf{current\_exns}\,:=\,e::\textsf{current\_exns}\,]
    }
  \end{array}
}{
}
\]
This rule introduces two new features, the \textsf{current\_exns}
field of the program state, and the \textsf{ClearExn} statement-form.
Both are present to support the ability of programs to use
\texttt{throw} without an argument to re-throw the ``current
exception''.  This is covered below in Section~\ref{sec:throw-none}.

Otherwise, the behaviour is clear: if the top handler has
``\texttt{...}'' as its parameter, then this handler is entered (and
the other handlers are discarded).

When the top handler has an explicitly-typed parameter, the handler is
only entered if the type of the thrown value matches
\ruleid{catch-specific-type-matches}:
\[
\infer{
  \begin{array}{l}
    \statebrack{
      \textsf{St}(\texttt{Catch}\;
                    (\texttt{throw}(e))\;
                    ((\lfloor(\mathit{pnm?},\tau)\rfloor, \mathit{hnd})::
                     \mathit{hnds}),c),
      \sigma
    }
    \to\\[.5ex]
    \quad \left\langle
      \begin{array}{l}
        \textsf{St}(
           \mathbf{\{}\,[\textsf{VDecInit}\;\tau\;\mathit{pnm}\;
                                            (\textsf{CopyInit}\;e)]\;
                        [\mathit{hnd};\,\textsf{ClearExn}]\,\mathbf{\}}, c),\\
        \sigma[\textsf{current\_exns}\,:=\,e::\textsf{current\_exns}\,]
      \end{array}\right\rangle
  \end{array}
}{
  \begin{array}{c}
  \mathit{pnm} = (\textsf{case}\;\mathit{pnm?}\;\textsf{of}\;*\to \texttt{" no name "}
  \;\;|\;\; \lfloor x \rfloor \to x)\\[.5ex]
  \textsf{exn\_param\_matches}\;\sigma\;\tau\;(\textsf{typeof}(e))
  \end{array}
}
\]
The string \texttt{" no name "} is chosen arbitrarily as the name of
the invisible temporary if the handler has just a type as its
parameter and no associated name.  This name is chosen so as to not
mask any existing names in scope (no legal \cpp{} program can have
variable names that include spaces).

If the declared type $\tau$ matches the type of the exception value,
then the exception value copy-initialises the parameter, and the
handler body is executed.  The constant \textsf{exn\_param\_matches}
checks the match, embodying the rules in \S15.3, paragraph 3.  If
there is no match, then the remaining handlers have to be tried
\ruleid{catch-specific-type-nomatch}:
\[
\infer{
  \begin{array}{c}
    \statebrack{
      \textsf{St}(\texttt{Catch}\;
                    (\texttt{throw}(e))\;
                    ((\lfloor(\mathit{pnm?},\tau)\rfloor, \mathit{hnd})::
                     \mathit{hnds}), c),
      \sigma
    }\\
    \to\\
    \quad\statebrack{
      \textsf{St}(\texttt{Catch}\;(\texttt{throw}(e))\;\mathit{hnds},c),
      \sigma
    }
  \end{array}
}{
  \lnot\,\textsf{exn\_param\_matches}\;\sigma\;\tau\;(\textsf{typeof}(e))
}
\]
If no handlers, remain, the exception propagates further
\ruleid{catch-stmt-empty-hnds} (generalised to allow any statements to
pass through):
\[
\infer{
  \statebrack{
    \textsf{St}(\texttt{Catch}\;s\;[\,], c),
    \sigma
  }
  \to
  \statebrack{\textsf{St}(s, c), \sigma}
}{\rule{0mm}{2ex}}
\]


\subsubsection{Using \texttt{throw} with No Argument}
\label{sec:throw-none}

If flow of control is within an exception handler, or within a
function body that has been called from such, then it is permissible
to use the expression \texttt{throw} without any arguments to cause
the current exception to be rethrown.  This requires the model to
track what the current handled exception is.  More, the
standard requires the state to track the notion of ``most recently
caught'' exception~(\S15.1, paragraph 7), which requires the state to
track a stack of exceptions that have been caught.

The expression version \ethrow{} is converted to the statement form as
soon as it is encountered \ruleid{expression-throw-none}:
\[
\infer{\statebrack{\textsf{Ex}(\ethrow(*),\mathit{se}), \sigma}
  \to
  \statebrack{\textsf{St}(\texttt{throw}(*),c), \sigma}
}{}
\] (The choice of $c$ is again irrelevant.)

There are then two rules for the statement $\texttt{throw}(*)$.  If
there is a current exception, all is well
\ruleid{bare-throw-succeeds}:
\[
\infer{
  \statebrack{\textsf{St}(\texttt{throw}(*), c), \sigma}
  \to
  \statebrack{\textsf{St}(\texttt{throw}(e), c),
    \sigma[\textsf{current\_exns} := \mathit{es}]}
}{
  \sigma.\textsf{current\_exns} = e :: \mathit{es}
}
\]
Otherwise, the program must call the \texttt{std::terminate} function
\[
\infer{
  \statebrack{\textsf{St}(\texttt{throw}(*), c), \sigma}
  \to
  \statebrack{\textsf{St}(\texttt{std::terminate}(), c),\sigma}
}{
  \sigma.\textsf{current\_exn} = [\,]
}
\]

Above, the rules for handlers also use a statement-form
\textsf{ClearExn}.  This special value has the following rule
\ruleid{clear-exn}:
\[
\infer{
\statebrack{\textsf{St}(\textsf{ClearExn},c), \sigma}
\to
\statebrack{\textsf{St}(\textsf{EmptyStmt},c),
  \sigma[\textsf{current\_exns}:=\mathit{es}]}
}{
  \sigma.\textsf{current\_exns} = e::es
}
\]
This ensures that when a handler finishes the most recently caught
exception is no longer recorded as such.  If a handler rethrows the
current exception, or throws a fresh exception of its own, and this
exception escapes the handler, then the flow of control will never
reach the \textsf{ClearExn}, and this rule will not fire.

\subsubsection{Exception Specifications}

The standard's \S15.4 specifies a method whereby functions can specify
which exception types they will produce.  If an unexpected exception
value occurs, this results in a call to \texttt{std::unexpected}.
This is not modelled in the dynamic semantics as it can be emulated
with a compile-time rewriting of the program.  If a function
\texttt{f} is specified to only raise exceptions X and Y, then it can
be rewritten to be
\begin{verbatim}
   f(args)
   {
     try {
       body
     }
     catch (X) { throw; }
     catch (Y) { throw; }
     catch (...) { std::unexpected(); }
   }
\end{verbatim}

\subsubsection{Exceptions and Object Lifetimes}

Exceptions complicate the story about the construction and destruction
of objects.  The system described in Deliverable~3 carefully arranged
things so that class objects, and their sub-objects were all
``created'' in the outermost scope, at the level where the constructor
for the outermost object appeared.  This meant that when the enclosing
block finished, the destructors for all of those objects and
sub-objects would be called as appropriate.

Unfortunately, this does not serve so well if constructors are allowed
to raise exceptions.  If this occurs, then any sub-objects that have
already been constructed have to have their destructors called
immediately.  Consider for example
\begin{alltt}
  Class::Class(objty p1) : b1(3), b2(\(e\)) \{ \(\mathit{body}\) \}
\end{alltt}
There are three objects that have their constructors called as a
result of a call to this constructor.  One is the parameter
\texttt{p1}, and the other are the base classes (or data-members)
\texttt{b1} and \texttt{b2}.  When this constructor is called,
\texttt{p1} is always destroyed, but (sub-)objects \texttt{b1} and
\texttt{b2} should live on, unless $\mathit{body}$ or $e$ cause an
exception to be raised.

To get this situation to work in the model, the state's
\texttt{blockclasses} field has to record two sorts of object
construction (which are to be unwound later), sub-object creation and
normal object creation.  When a block exits normally, sub-object
destructors are not called, but instead delayed so that they can be
recorded as having the same lifetime as the enclosing object.  In
Deliverable~3's version of the semantics, all sub-objects were
constructed at the same, uppermost, level, but the addition of
exceptions requires the model to be able to destroy sub-objects
earlier.

\subsection{Templates}
\label{sec:templates}

In this section, I describe how the semantics models templates.  I
have been inspired by Siek and Taha~\cite{DBLP:conf/ecoop/SiekT06},
though as I shall discuss, the dynamics of their model is too
simplistic for the full language of templates, in particular handling
template parameters that are themselves references to templates
(``higher order templates'' if you will).\footnote{Siek and Taha do
  model \texttt{typedef} declarations within classes, which I do not.}

Previously, I suggested that it might be possible to treat templates
at ``run-time'', as if one had written a one-pass \cpp{} interpreter.
However, the example in Figure~\ref{fig:templates-not-interpretable}
demonstrates that such a goal is impossible, that one would have to
write a two pass interpreter at the very least.  Any reference to the
figure's \texttt{List} template, perhaps within a function that was
called at a great stack depth, causes the need to statically
initialise the global \texttt{node\_count} before the program even
begins.  A \naive{} interpreter that attempts to execute the program
source as it sees it, will come unstuck.

As the interpreter sees the template declarations above, it does
nothing.  Then it performs its global declarations, and jumps into
main.  At this point it has already failed to do the right thing.
Instead, the putative interpreter would have to scan the whole program
for template applications so that it can generate the appropriate
global variable initialisations.  This is no better than explicitly
pre-compiling templates, so I have adopted an explicit two-phase
compilation approach.

\begin{figure}
\begin{verbatim}
  template <class T> class List {
    T item;
    List *next;
    static int node_count;
  };

  template<class T> int List<T>::node_count = 0;
\end{verbatim}
\caption{A program demonstrating the difficulty of interpreting
  templates.}
\label{fig:templates-not-interpretable}
\end{figure}


\subsubsection{Names and Identifiers}
Modelling templates raises a number of interesting questions.  One of
the more trivial seeming issues is the representation of types and
names when the former can have the latter embedded within them.
Indeed, names end up needing to be mutually recursive with types
(which has the follow-on effect of eliminating the \texttt{names}
theory from the development; names are now defined as part of the
\texttt{types} theory).

\newcommand{\bnfof}{\;\textsf{of}\;}
\newcommand{\bnfcurry}{\;\texttt{=>}\;}
\newcommand{\stringty}{\textsf{string}}
\newcommand{\boolty}{\textsf{bool}}
\newcommand{\intty}{\textsf{int}}
The new definitions begin by defining the notion of ``top-level
name'' (\texttt{TopName}) below.  This is the sort of name that can
occur with namespace components, possibly with a leading \texttt{::}
to indicate that the ``path'' is an absolute one.
\[
\texttt{TopName} \;\;=\;\; \boolty \times \stringty^*
\times \stringty
\]
If true, the boolean field indicates that this is a name prefixed by
the \texttt{::} operator.  The list of names are those of enclosing
namespaces.  For example, \texttt{ns1::subspace::foo} is represented
by the triple \[(\textsf{false}, [\texttt{"ns1"}, \texttt{"subspace"}],
\texttt{"foo"})\]

Next, the template identifiers can be specified:
\[
\texttt{TemplateID}
\begin{array}[t]{cl}
  ::= & \texttt{TemplateConstant} \bnfof \texttt{TopName}\\
  |   & \texttt{TemplateVar}      \bnfof \stringty
\end{array}
\]

\newcommand{\cppid}{\texttt{CPP\_ID}}
\bigskip Now we are ready to specify types.  Types need to include
identifiers (\cppid{} henceforth) directly in order to model
something like
\begin{alltt}
  template <class T> int size<T>(T x) { return x.size(); }
\end{alltt}
where \texttt{T} has to be viewed as a type in the parameter list of
the function.

So, we extend the type
of types, by allowing identifiers:
\[ \texttt{CPP\_Type} \begin{array}[t]{cl}
  ::= & \texttt{int} \;\;|\;\; \texttt{char} \;\;|\;\; \texttt{float} \\
| & \dots\\
| & \texttt{TypeID}\bnfof \cppid
\end{array}
\]
Identifiers may be template calls, so that it is legitimate to write
\texttt{f<myclass>} or \texttt{List<T>}, and identifiers can also
refer to sub-classes and (static) fields of classes.  This means that
we must define \cppid{} thus:
\[\cppid \begin{array}[t]{cl}
  ::= & \texttt{IDVar}\bnfof\stringty\\
| & \texttt{IDConstant} \bnfof\texttt{TopName}\\
| &
\texttt{IDTempCall}\bnfof\texttt{TemplateID}\bnfcurry \texttt{TemplateArg}^*\\
| & \texttt{IDFld}\bnfof\cppid\bnfcurry\texttt{StaticField}
\end{array}
\]
Recall that \texttt{=>} is the HOL notation for specifying that the
arguments to a constructor are curried.  Thus, for example, we write
$\texttt{IDFld}\;\mathit{id}\;\mathit{sfld}$.

Fields within classes can themselves be templates (template member
functions), so the \texttt{StaticField} type is defined
\[
\texttt{StaticField} \begin{array}[t]{cl}
  ::= &
  \texttt{SFTempCall}\bnfof\stringty\bnfcurry\texttt{TemplateArg}^*\\
| & \texttt{SFName}\bnfof\stringty
\end{array}
\]
Finally, we need to specify the possible sorts of arguments that can
be passed to templates.  The Standard is quite explicit
here~\cite[\S14.3~para~1]{cpp-standard-iso14882}, there are three
sorts of arguments: types, templates, and ``non-type, non-template''
arguments (meaning references to objects with linkage, or numbers).
Thus:
\[
  \texttt{TemplateArg} \begin{array}[t]{cl}
::= & \texttt{TType}\bnfof\texttt{CPP\_Type}\\
|   & \texttt{TTemp}\bnfof\texttt{TemplateID}\\
|   & \texttt{TVal}\bnfof\texttt{TemplateValueArg}
\end{array}
\]
Finally, there are four different sorts of non-type, non-template
arguments~\cite[\S14.3.2~para~1]{cpp-standard-iso14882}:
\[\texttt{TemplateValueArg}
\begin{array}[t]{cl}
::= & \texttt{TNum}\bnfof\intty\\
|   & \texttt{TObj}\bnfof\cppid\\
|   & \texttt{TMPtr}\bnfof\cppid\bnfcurry\texttt{CPP\_Type}\\
|   & \texttt{TVAVar}\bnfof\stringty
\end{array}
\]
This presentation is slightly simplified because the standard also
allows arithmetic on these arguments, where this is appropriate
(between \texttt{TNum} and \texttt{TVAVar} parameters).

This is a complicated type, and it does assume that some previous
phase of analysis has determined which references are to class types,
and which are to namespaces.  In this way, something like
\texttt{x::y::z} can be split up to include namespaces within a
\texttt{TopName} component, and sub-fields or sub-classes within
\texttt{IDFld} structuring.

Similarly, a string such as \texttt{x::t<int>} is ambiguous without
some static analysis being performed to determine whether \texttt{x}
is a namespace or a class.  If \texttt{x} is a class, then \texttt{t}
must be a template member function, but if it is a namespace, then
\texttt{t} might be either a class template or a function template.

Further note that there can be at most two template calls within an
identifier.  If there are two, then the outermost is a class template,
and the second will occur last, and will be a template member
function.

\subsubsection{Instantiation and Matching}

(This seection describes formalisation done in the script file
\texttt{instantiation}.)

\medskip
\noindent Types and identifiers can be \emph{instantiated}: mappings
from variable names to values are applied over the structure of the
value (type or identifier), and occurrences of variable names are
replaced by the appropriate element from the range of the function.
Because there are three sorts of variables (corresponding to the three
different sorts of template argument), an instantiation is actually a
triple of functions (one for each sort of variable).

In Siek and Taha~\cite{DBLP:conf/ecoop/SiekT06}, instantiation is a
very elegant operation.  In a more faithful model of more of \cpp,
more complexities intrude.  In addition to the need for three
mappings, the model must also accept that instantiation can result in
an invalid result.  Instantiation must become partial, which is
modelled by making the types of the various instantiation functions be
of the form
\[
\texttt{inst<}\tau\texttt{>} : \mathit{substitution} \rightarrow \tau
\rightarrow \tau\;\textsf{option}
\]
The partiality arises at the lowest level, as in the following
example:
\begin{verbatim}
   template<class T> void f<T>(int x) { T::staticfield = x; }
   void g() { f<int>(3); }
\end{verbatim}
This must be an error because it is not sensible to write
\texttt{int::staticfield}.  (Other type substitutions may also cause
this to be an error, but this error can be detected as the
substitution is done, without any need to lookup information about the
argument.)

The partiality of instantiation does not prevent us from defining a
partial order over types, such that $\tau_1 \leq \tau_2$ when $\tau_2$
is a more specialised/instantiated version of $\tau_1$.  As in Siek
and Taha~\cite{DBLP:conf/ecoop/SiekT06}, we can prove reflexivity,
transitivity, and antisymmetry (up to renaming of free variables).

Given the partial order, it is straightforward to find the best match
amongst a set of template definitions for a given template call.

\subsubsection{Program Instantiation}

Siek and Taha have an elegant model for program instantiation.  A
program is a sequence of definitions (of classes, and of static member
functions).  A definition may cause an existing template to be
instantiated because of a reference to that template within the
definition.  When a member function definition is encountered, if its
body includes a reference to other functions, these functions may need
to be instantiated.

For example, when analysing the program in Figure~\ref{fig:taha-prog},
the Siek and Taha's model will see the reference to
\texttt{Foo<T*>::f()} in the definition of \texttt{Bar<T>::g()} and
instantiate the definition of \texttt{f} (it knows that it does not
already have an instantiation for a type of the form \texttt{T*}).
This instantiation will result in a template definition (one with free
variables), which may or may not be required in the rest of the
program.
\begin{figure}
\begin{verbatim}
   template <class T> class Foo { static int f(); };
   template <class T> int Foo<T>::f() { return 3; }

   template <class T> class Bar { static int g(); };
   template <class T> int Bar<T>::g() { return Foo<T*>::f(); }
\end{verbatim}
  \caption{In Siek and Taha's model, the definition of
    \texttt{Foo<T>::f()} will get instantiated to provide a definition
    of \texttt{Foo<T*>::f()}.}
\label{fig:taha-prog}
\end{figure}

This model breaks down in the presence of template parameters that are
templates because it becomes impossible to determine the dependencies
of a template definition.  In the program in
Figure~\ref{fig:taha-problem}, it is impossible to tell what
definition should be instantiated when processing the definition of
\texttt{Baz<A>::g}.  In the presence of template parameters, Siek and
Taha's model is too eager.

\begin{figure}
\begin{verbatim}
   template <class T> struct Foo { static int f(); };
   template <class T> int Foo<T>::f() { return 3; }

   template <class T> struct Bar { static int f(); };
   template <class T> int Bar<T>::f() { return 4; }

   template <template <class> class A> struct Baz {
     static int g();
   };
   template <template <class> class A>
   int Baz<A>::g() {
     return A<int>::f();
   }

   int main() { return Baz<Foo>::g(); }
\end{verbatim}
  \caption{Siek and Taha's model's early instantiation breaks down
    when it sees the definition of \texttt{Baz<A>::g}.  At this point,
    it can not tell which template is being instantiated in the body.
    By way of contrast, my model doesn't instantiate anything until it
    sees the definition of \texttt{main}.
    (This program is available as
    \texttt{notes/siek-taha-tempvar.cpp})}
\label{fig:taha-problem}
\end{figure}

My model only performs instantiations when there is a ground instance
to drive the instantiation.  Otherwise, it is similar to what is
presented in Siek and Taha.  In particular, it is important to avoid
instantiating member functions unless they are called for.

The model in the \texttt{templates} script is based around a working
state of four components: Templates, Residuals, Needs and
Declarations.  The Templates are those declarations encountered so far
which are of templates.  The Residuals are those ``resolved''
declarations that have either been encountered, or which have been
instantiated from Templates, and are thus ``ground'' or variable-free.
A resolved declaration is one that has had all of its dependencies
appropriately instantiated, or at least recorded.  The Needs are those
declarations that have been required for a resolution, and for which
there have been template declarations, but no definitions.  Needs can
be functions (class-members, including constructors and destructors,
or normal top-level functions), and static data members of classes.
Finally, the Declarations are the declarations that still need to be
processed, coupled with a number specifying how much work has been
done to the declaration so far (all declarations start at level 0).

The Declarations component of the instantation state can grow as well
as shrink, and it is this that can cause compilation to fail to
terminate.

Values in the Declarations component can be declarations or
definitions, and can be of classes, variables and functions.  The
behaviour on encountering declarations of ground entities is rather
involved, and detailed in Table~\ref{tab:ground-decls}.  The basic
idea is that any ground definition can cause the instantiation of
templates.  Such instantiations are always governed by the requirement
that the best match is found.  Successful instantiations are stored in
the Residuals component, so duplicate work is avoided.

Instantiations are done in two phases.  First any template classes
required are instantiated.  This instantiation can in turn prompt
further instantiation because a template class can have template
members, or template ancestors.  if the required types can be
instantiated without causing an infinite loop, the original definition
will eventually return to the top of the Declarations list, but at
Level 1.  This time, the definition is scanned to see if it depends on
any functions or static variables.   These references may or may not
have definitions (though they will all at least have declarations).
Those with definitions are instantiated and put onto the list of
Declarations.  Those without end up in Needs.

When a template declaration is encountered, this is put straight into
the Templates component.  When a template definition is encountered,
this may resolve an outstanding Need, so it is checked against all
current Needs.  Those that can instantiate against the new Template
definition are put onto the Declarations list.

\begin{table}
\hspace{-3em}
\begin{tabular}{lp{0.32\textwidth}p{0.32\textwidth}p{0.32\textwidth}}
  & \multicolumn{1}{c}{\textbf{Function}} &
  \multicolumn{1}{c}{\textbf{Class}} &
  \multicolumn{1}{c}{\textbf{Variable}} \\
  \textbf{Decl.}
  &
  \small \textbf{L0:}
  Add declarations of any referred to template types
  to the list of Declarations still to be processed.
  &
  % Class
  \small \textbf{L0:}
  If not already in Residuals, and a template application, find best
  instantiation for this type and add its instantiated definition
  (less any member function definitions) to
  list of Declarations still to be processed.
  &
  \small \textbf{L0:} Declarations can only be of non-template variables
  (template variables are declared inside class definitions).  Add to
  Residuals and continue.
  \\ \\
  \textbf{Defn.}
  &
  % Function 0
  \small\textbf{L0:} If not already present in Residuals, extract any
  template
  types, and put declarations of
  these into Declarations, ahead of this function definition at
  Level~1.
  &
  % Class 0
  \small \textbf{L0:} Extract any template types from the data-member
  definitions, and put Level~0 declarations for these into
  Declarations, followed by this declaration at Level~1.
  &
  % Variable 0
  \small \textbf{L0:} Add any referenced template types to
  declarations at Level~0 (such types may occur in the initialising
  expression for the variable), followed by the same declaration at
  Level~1.
  \\[1ex]
  &
  % Function 1
  \small \textbf{L1:} Extract any function and static
  variable references, and
  add those with definitions (once instantiated with best match)  to
  Declarations at Level 0.  Those
  functions or static variables without definitions are added to
  Needs.
  &
  % Class 1
  \small \textbf{L1:}
  Required types have been dealt with: now check for definitions of
  any static data members.  Those not present are added to Needs.
  Best matched members are added to Declarations at Level~0.
  &
  % Variable 1
  \small \textbf{L1:} Check initialising expression for references to
  template functions, or other static variables.  Add instantiated
  definitions (for those with definitions) to
  Declarations list at Level~0.  Add others to Needs.
\end{tabular}
\caption{Behaviour of program instantiation on ground declarations}
\label{tab:ground-decls}
\end{table}


\bigskip
\noindent
Unlike Siek and Taha's model, there hasn't been any validation of my
model of program instantiation, whether by the proof of theorems, or
by simulation of concrete examples.  I hope to address this in the
final deliverable.


\subsubsection{Resolving Names}

One significant issue, at least as far as the standard is concerned,
is the resolution of names in templates.  When a template definition
is instantiated, it is important to specify how the names occuring in
the template bind.  Typically, such names might bind to global names
that are in scope at the point of the template's definition, or to
member functions associated with the template argument.

The rule is actually fairly straightforward, at least in principle:
names are bound to argument names when the statically determined type
of the name refers to the template parameter.  I do not model this
issue as static processing can resolve it before compilation begins.
In particular, if a name is determined to depend on a template
parameter, then that name can be prefixed with an explicit use of
\texttt{this->}, making it clear that a member is being referred to.

\section{Future}

The final deliverable calls for sanity theorems, namespaces, run-time
type identification and library functions.  All of these save the last
are strightforward.  It is now not clear to me what is meant by
``library functions''; whether actual functions are to be specified,
or whether this is a reference to the mechanisms for linking to
library functions (paragraph three of \S3 of the Statement of Work).

\bibliographystyle{plain}
\bibliography{deliverables}

\end{document}

%%% Local Variables:
%%% mode: latex
%%% TeX-master: t
%%% End:
