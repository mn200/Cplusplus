\documentclass[11pt]{article}

\usepackage{charter}
\usepackage{alltt}
\usepackage{url}
\usepackage{proof}
\usepackage{amsfonts}

\newcommand{\cpp}{\mbox{C\hspace{-0.5pt}+\hspace{-1.5pt}+}}
\newcommand{\ruleid}[1]{\textsl{[#1]}}

\newcommand\ckey[1]{\texttt{#1}}

\newcommand\statebrack[1]{\langle #1 \rangle}
\newcommand\exprarrow{\rightarrow_e}
\newcommand\tcexarrow{\exprarrow^*}
\newcommand\expreval[2]{\statebrack{#1} \exprarrow \statebrack{#2}}
\newcommand\tcexeval[2]{\statebrack{#1} \tcexarrow \statebrack{#2}}
\newcommand\stmtarrow{\rightarrow_s}
\newcommand\stmteval[2]{\statebrack{#1} \stmtarrow \statebrack{#2}}
\newcommand\vardarrow{\rightarrow_v}
\newcommand\vardeval[2]{\statebrack{#1}\vardarrow \statebrack{#2}}
\newcommand\hoaretriple[3]{\{#1\}\;#2\;\{#3\}}
\newcommand\hoarequad[4]{\{#1\}\;#2\;\{#3\}\{#4\}}

\newcommand\inferwhere[1]{#1\par\nopagebreak\vskip 1mm\nopagebreak%
                          \noindent\ignorespaces}

\newcommand\okval{\ckey{okVal}}
\newcommand\tyobj[1]{\operatorname{\textbf{obj}}[#1]}

\newcommand\addse{\ckey{add\_se}}
\newcommand\applyse{\ckey{apply\_se}}
\newcommand\clearse{\ckey{clear\_se}}
\newcommand\declvar{\ckey{decl\_var}}
\newcommand\initdeclvar{\ckey{init\_decl\_var}}
\newcommand\installparams{\ckey{inst\_parms}}
\newcommand\markref{\ckey{mark\_ref}}
\newcommand\memtrans{\ckey{mem\_trans}}
\newcommand\removerefs{\ckey{remove\_refs}}
\newcommand\structdeclare{\ckey{struct\_decl}}

\newcommand\cid{{\sf id}}
\newcommand\cgenop{\odot}
\newcommand\Cand{\ckey{\&\&}}
\newcommand\Cor{\ckey{||}}
\newcommand\ccond[3]{#1 \;\ckey{?}\; #2 \;\ckey{:}\; #3}
\newcommand\cassign[2]{#1\mathbin{\ckey{=}}#2}
\newcommand\ccast[2]{\ckey{(}#1\ckey{)}#2}
\newcommand\copassign[3]{#1\mathbin{#2\ckey{=}}#3}
\newcommand\andorsqpt{\widehat{\ckey{\&|}}}
\newcommand\cptr{\ptrtype}
\newcommand\se[2]{\clubsuit(#1,#2)}
\newcommand\rhsop{\diamondsuit}
\newcommand\ass[4]{%
  #2 \mathbin{\raisebox{-1.5mm}{$\stackrel{#4}{#1\ckey{=}}$}} #3}
\newcommand\cif[3]{%
\mbox{\ckey{if} \ckey{(}$#1$\ckey{)} $#2$ \ckey{else} $#3$}}
\newcommand\cloop{\mbox{$O$}}
\newcommand\trap{\mbox{$T$}}

\newcommand\declstmts{\mbox{\sl decls}\;\mbox{\sl stmts}}
\newcommand\cblock[1]{\mbox{\bf \{} #1\mbox{\bf \}}}
\newcommand\cwhile[2]{\mbox{\ckey{while} \ckey{(}$#1$\ckey{)} $#2$}}
\newcommand\cfor[4]{\mbox{\ckey{for ($#1$; $#2$; $#3$) $#4$}}}
\newcommand\cdowhile[2]{\mbox{\ckey{do $#1$ while ($#2$);}}}
\newcommand\tradwhile[2]{\mbox{\bf while}\;#1\;\mbox{\bf do}\;#2}
\newcommand\tradif[3]{\mbox{\bf if}\;#1\;\mbox{\bf then}\;#2\;%
                      \mbox{\bf else}\;#3}

\newcommand\ptrtype[1]{#1\ckey{*}}
\newcommand\arraytype[2]{#1\ckey{[}#2\ckey{]}}
\newcommand\sgndtype[1]{\ckey{signed}\;#1}
\newcommand\usgndtype[1]{\ckey{unsigned}\;#1}
\newcommand\void{\ckey{void}}
\newcommand\longdouble{\ckey{long~double}}
\newcommand\structtype[1]{\ckey{struct}\;#1}
\newcommand\struct{\ckey{struct}}

\newcommand\cnum{\ckey{Num}\,}
\newcommand\cfunref{\ckey{Funref}\,}
\newcommand\cnull{0}
\newcommand\cvar{\ckey{Var}\,}
\newcommand\caddr[1]{\ckey{\&}#1}
\newcommand\cderef{\ckey{*}\,}
\newcommand\cbreak{\ckey{break}}

\newcommand\cholundef{{\cal U}}
\newcommand\cvalue[2]{\underline{(#1,#2)}}
\newcommand\clvalue{\ckey{LV}}
\newcommand\cfvalue{\ckey{FV}}
\newcommand\crvreq{\ckey{RVR}}


\newcommand\retval{\ckey{RetVal}}
\newcommand\stmtval{\ckey{StmtVal}}
\newcommand\breakval{\ckey{BreakVal}}
\newcommand\contval{\ckey{ContVal}}

\newcommand\vardval{\ckey{VarDeclVal}}


\newcommand\wftype[2]{#1\vdash \text{\emph{wf}}(#2)}
\newcommand\etyping[3]{#1\vdash #2 : #3}

\newcommand\ctilde{\ckey{\~{}}}

\newcommand\lvtorv{\leadsto}
\newcommand{\funtype}[2]{#1 \rightarrow #2}
\newcommand{\SOME}[1]{\lfloor #1 \rfloor}

\newcommand{\last}{\textsf{last}}
\newcommand{\hd}{\textsf{hd}}

\newenvironment{proof}{\sl\setlength\parindent{0mm}\addtolength\parskip{2mm}}%
{}

% Local Variables:
% mode: latex
% TeX-master: t
% End:


\title{Notes on Deliverable 3 (August 2006)}
\author{Michael Norrish\\{\small \texttt{Michael.Norrish@nicta.com.au}}}
\date{}


\begin{document}
\maketitle

\section{Form}

This deliverable consists of a compressed \texttt{tar}-file, that when
unpacked consists of a directory called \texttt{qinetiq-cpp}, which in
turn contains four directories
\begin{itemize}
\item \texttt{holsrcs}, containing the HOL source files of the
  mechanisation.  These files will build with the version of HOL4
  present in the CVS repository at SourceForge, with timestamp
  \texttt{2006-07-31 00:00Z}.  See Section~\ref{sec:getting-hol}
  below for instructions on how this version of HOL can be retrieved,
  and how the deliverable's HOL source files can then be built and
  checked.
\item \texttt{talks}, containing the \LaTeX{} source and a PDF for the
  talk presented at the DARP meeting in Newcastle in April 2006.  The
  source assumes that the \texttt{latex-beamer} and \texttt{PSTricks}
  packages are available.
\item \texttt{docs}, containing \LaTeX{} sources and a PDF version of
  this document, as well as sources for the notes on Deliverable~1.
\item \texttt{notes}, some C++ source files that illustrate various
  aspects of C++ behaviour.  An accompanying text file explains some
  of the behaviours.
\end{itemize}

\subsection{Building HOL Source-Files}
\label{sec:getting-hol}

\paragraph{Getting HOL From SourceForge}

To get a particular, dated, version of the HOL4 sources from the CVS
repository, one must first issue the command

{\small
\begin{verbatim}
   cvs -d:pserver:anonymous@hol.cvs.sourceforge.net:/cvsroot/hol login
\end{verbatim}
}

When prompted for a password, just press \texttt{Enter} to send a null
response.  The check-out of source code from SourceForge can now
proceed.  The source code fits into 60MB.  Issue the command

{\small
\begin{alltt}
   cvs \textit{repository-spec} co -D \textit{date} hol98
\end{alltt}
}

\noindent where \textit{\ttfamily repository-spec} is the string

{\small
\begin{alltt}
   -d:pserver:anonymous@hol.cvs.sourceforge.net:/cvsroot/hol
\end{alltt}
}

\noindent (also used in the \texttt{login} command), and where
\textit{\ttfamily date} is the desired date, best specified as an
ISO~8601 string enclosed in double-quotes.  For example,
\texttt{"2006-06-30 04:05Z"}.

Once a copy of the sources have been downloaded, further commands can
be used to update this copy to correspond to different dates.  As long
as such commands are issued from within the \texttt{hol98} directory,
the repository specification can be omitted.  The update command is

{\small
\begin{alltt}
   cvs update -d -D \textit{date}
\end{alltt}
}

\paragraph{Installing HOL} Once the sources have been downloaded, the
installation instructions from the page at
\url{http://hol.sourceforge.net} should be followed to build a copy of
HOL.  An installation of the Moscow~ML compiler (v2.01) will also be
required.

\paragraph{Building Deliverable Sources}
When HOL4 has been installed, the \texttt{Holmake} program (found in
the \texttt{hol98/bin} directory) can be run in the \texttt{holsrcs}
directory to create and check the logical theories.

\section{Content}

This deliverable extends the existing HOL semantics for deliverable
2's ``\cpp-like'' language by adding treatments of \texttt{const},
multiple inheritance and constructor and destructor functions.  In
addition a number of other wrinkles are addressed.  The existence of
the latter suggests that a more detailed plan of work for other
language features should be drawn up.

\subsection{Postponing Statics}

There is a great deal of language in the standard encoding complicated
rules that are entirely static in their scope.  In this context,
``static'' means that these are rules that will be implemented by
compilers, and that programs that fail to be adhered to them will be
flagged as such at compile time.

For example, one such rule is the requirement that data members in
classes have unique names.  (Multiple functions can share the same
name because of overloading.)

\section{Future}

The next deliverable will include a treatment of multiple inheritance,
\texttt{const} types and object lifetimes.  Multiple inheritance
should be straight-forward, given the Wasserrab's model.  The
treatment of \texttt{const} should also be fairly easy.  Object
lifetimes will require a treatment of class constructors and
destructors at the very least, and could well be moderately
complicated, particularly if the standard's level of
under-specification is to be modelled.

If destructors are to be called before a certain deadline is reached,
this might best be modelled in a way similar to the treatment of side
effects.  Object destructions can be accumulated in some sort of state
field (a set, say), and non-deterministically pulled from this as
evaluation progresses.  When a deadline is reached, there will be a
side-condition on the relevant rule requiring that the state field be
empty.

\bibliographystyle{plain}
\bibliography{deliverables}

\end{document}


%%% Local Variables:
%%% mode: latex
%%% TeX-master: t
%%% End:


%%% Local Variables:
%%% mode: latex
%%% TeX-master: t
%%% End:
